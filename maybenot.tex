%!TEX program = xelatex
\pdfminorversion=4 % required for impressive presentation tool
\documentclass[xcolor=x11names,dvipsnames,aspectratio=169]{beamer}
\usepackage{graphicx}
\usepackage{textcomp}
\usepackage{flowchart}
\usepackage[absolute,overlay]{textpos}
\usepackage{tikz}

\usetikzlibrary{
  shapes.misc,positioning,arrows,decorations.markings,shapes.arrows
}
\tikzset{set/.style={draw,circle,inner sep=0pt,align=center}}
\tikzset{brace/.style={decorate, decoration={brace}},
  brace mirrored/.style={decorate, decoration={brace,mirror}}}
	\pgfdeclarelayer{background}
	\pgfsetlayers{background,main}

  \usepackage[
    lambda, advantage, operators, sets, adversary, landau, probability, notions,
    logic, ff, mm, primitives, events, complexity, asymptotics, keys
  ]{cryptocode}

\usetheme{Dart}

\definecolor{rust}{RGB}{222,165,132}
\definecolor{tor}{RGB}{148,0,255}

\title{Maybenot}
\subtitle{A framework for traffic analysis defenses}
\author{\href{https://pulls.name}{Tobias Pulls} and \href{https://www.ethanwitwer.com/}{Ethan Witwer}}
\date{26 November 2023}

\begin{document}
	\frame{\titlepage}

\setcounter{showSlideNumbers}{0}

\begin{frame}{Patterns in encrypted data}
  \framesubtitle{CC BY-SA 4.0 \url{https://words.filippo.io/the-ecb-penguin/}}
  \begin{center}
    \includegraphics<1>[width=.2\textwidth]{img/Tux}%
  \end{center}
\end{frame}

\begin{frame}{Patterns in encrypted data}
  \framesubtitle{CC BY-SA 4.0 \url{https://words.filippo.io/the-ecb-penguin/}}
  \begin{center}
    encrypted with AES-ECB\\
    \includegraphics<1>[width=.35\textwidth]{img/tux-ecb}%
  \end{center}
\end{frame}

\begin{frame}{Patterns in encrypted data}
  \framesubtitle{CC BY-SA 4.0 \url{https://words.filippo.io/the-ecb-penguin/}}
  \begin{center}
    encrypted with AES-CBC\\
    \includegraphics<1>[width=.35\textwidth]{img/tux-rng}%
  \end{center}
\end{frame}

\begin{frame}{Patterns in encrypted network traffic}
  \framesubtitle{from Tor, 514 byte cells}
  \begin{center}
    \includegraphics<1>[width=.9\textwidth]{img/trace-nopadding}%
  \end{center}
\end{frame}

\begin{frame}{Patterns in encrypted network traffic}
  \framesubtitle{from Tor, 514 byte cells}
\begin{columns}
  \begin{column}{0.4\textwidth}
    \begin{itemize}
      \item wikipedia.org, 10 articles
      \vspace{0.8cm}
      \item amazon.com, 10 products
      \vspace{0.8cm}
      \item reddit.com, 10 subreddits
      \vspace{0.8cm}
      \item okezone.com, 10 articles
      \vspace{0.8cm}
      \item yahoo.co.jp, 10 articles
    \end{itemize}
  \end{column}
  \begin{column}{0.6\textwidth}
    \begin{center}
      \includegraphics<1>[width=.9\textwidth]{img/trace-nopadding-cut}%
    \end{center}
  \end{column}
  \end{columns}
\end{frame}

\begin{frame}{Goal of Maybenot: increase attacker uncertainty \includegraphics<1>[width=.022\textwidth]{img/maybenot}}
  \framesubtitle{black/white = received/sent nonpadding, red/green = received/sent padding}
  \begin{columns}
    \begin{column}{0.5\textwidth}
      \begin{center}
        \includegraphics<1>[width=.9\textwidth]{img/trace-nopadding-cut}%
      \end{center}
    \end{column}
    \begin{column}{0.5\textwidth}
      \begin{center}
        \includegraphics<1>[width=.9\textwidth]{img/interspace}%
      \end{center}
    \end{column}
    \end{columns}
\end{frame}

\section{Maybenot framework}
\begin{frame}{Why a framework?}
  \framesubtitle{early days of deploying good-enough defenses}
  \begin{columns}
    \begin{column}{0.55\textwidth}
      \begin{itemize}
        \item perfect or \alert{good enough} defenses
        \item moving target -- advances in AI
        \item different types of traffic analysis attacks
        \item different protocols, similar needs
        \item \alert{orchestrate} defenses
      \end{itemize}
    \end{column}
    \begin{column}{0.45\textwidth}
      \begin{center}
        \includegraphics<1>[width=.45\textwidth]{img/setting-tls}\\
        \vspace{0.4cm}
        \includegraphics<1>[width=.65\textwidth]{img/setting-vpn}\\
        \vspace{0.4cm}
        \includegraphics<1>[width=.85\textwidth]{img/setting-tor}
      \end{center}
    \end{column}
    \end{columns}
\end{frame}

\begin{frame}{Maybenot framework}
  \framesubtitle{a framework for traffic analysis defenses}
  \begin{columns}
    \begin{column}{0.3\textwidth}
      \begin{center}
        \includegraphics<1>[width=.9\textwidth]{img/framework}%
      \end{center}
    \end{column}
    \begin{column}{0.7\textwidth}
      \begin{itemize}
        \item {a small light-weight library in \color{rust}Rust}
        \item generalization and extension of the {\color{tor}Tor Circuit
        Padding Framework} by Perry and Kadianakis
        \bigskip
        \item defenses as probabilistic state \alert{machines}
        \item with tunable \alert{limits} on padding and blocking
        \bigskip
        \item integration is simple
        \begin{description}
          \item[input] trigger \alert{events}
          \item[output] schedule \alert{actions}
        \end{description}
        \bigskip
        \item related work: QCSD by Smith et al., WFDefProxy by Gong et al.,
        Proteus by Wails et al.
      \end{itemize}
    \end{column}
    \end{columns}
\end{frame}

\section{Maybenot simulator}
\begin{frame}{Maybenot simulator}
  \framesubtitle{because collecting real-world datasets takes time}
  \begin{columns}
    \begin{column}{0.55\textwidth}
      \begin{itemize}
        \item another simple {\color{rust}Rust} library
        \item input
        \begin{itemize}
          \item base network trace
          \item machines at client and server
          \item simulation parameters: e.g., limits and network model
        \end{itemize}
        \item output: simulated network trace
        \item rapid \alert{prototyping} of defenses
      \end{itemize}
    \end{column}
    \begin{column}{0.5\textwidth}
      \begin{center}
        \includegraphics<1>[width=.99\textwidth]{img/sim-9-sites-criterion}%
      \end{center}
    \end{column}
    \end{columns}
\end{frame}

\section{Maybenot defenses}
\begin{frame}{Defense 1: FRONT}
  \framesubtitle{Jiajun Gong and Tao Wang. \emph{Zero-delay Lightweight Defenses against Website Fingerprinting}. USENIX Security 2020.}
  \begin{columns}
    \begin{column}{0.55\textwidth}
      \begin{itemize}
        \item a lightweight padding-based defense
        \item on each side of a connection:
        \begin{itemize}
          \item sample a padding count $n = [1, N]$
          \item sample a window $w = [W_{min}, W_{max}]$
          \item select $n$ time values from a Rayleigh distribution with $\sigma = w$
          \item send a packet at each time
        \end{itemize}
      \end{itemize}
    \end{column}
    \begin{column}{0.5\textwidth}
      \begin{center}
        \large{\alert{trace}} \\
        + \\
        \tiny{\hfill} \\
        \includegraphics<1>[width=.99\textwidth]{img/rayleigh.png}%
      \end{center}
    \end{column}
    \end{columns}
\end{frame}

\begin{frame}{Maybenot FRONT}
  \framesubtitle{a straightforward approach}
  \begin{center}
    \includegraphics<1>[width=.8\textwidth]{img/maybenot-front.png}%
  \end{center}
\end{frame}

\begin{frame}{Pipelined FRONT}
  \framesubtitle{towards a closer approximation}
  \begin{center}
    \includegraphics<1>[width=.8\textwidth]{img/maybenot-piplined-front.png}%
  \end{center}
\end{frame}
    
\begin{frame}{Defense 2: RegulaTor}
  \framesubtitle{James K Holland and Nicholas Hopper. \emph{RegulaTor: A Straightforward Website Fingerprinting Defense}. PETS 2022.}
  \begin{columns}
    \begin{column}{0.55\textwidth}
      \begin{itemize}
        \item regularization defense for Tor
        \item send traffic at a fixed rate $R$ initially, decrease via decay function: rate = $RD^t$
        \item if the number of queued cells exceeds a threshold, restart the surge (reset $t_0$)
        \item optimizations:
        \begin{itemize}
            \item padding budget
            \item $C$ parameter for upload traffic
        \end{itemize}
      \end{itemize}
    \end{column}
    \begin{column}{0.5\textwidth}
      \begin{center}
        \includegraphics<1>[width=.99\textwidth]{img/regulator-defended.png}%
      \end{center}
    \end{column}
    \end{columns}
\end{frame}

\begin{frame}{Maybenot RegulaTor}
  \framesubtitle{the relay-side machine}
  \begin{center}
    \includegraphics<1>[width=.6\textwidth]{img/maybenot-regulator.png}%
  \end{center}
\end{frame}

\begin{frame}{Maybenot RegulaTor}
  \framesubtitle{the client-side machine}
  \begin{center}
    \includegraphics<1>[width=.4\textwidth]{img/maybenot-regulator-client.png}%
  \end{center}
\end{frame}

\begin{frame}{Evaluation results}
  \framesubtitle{Tobias Pulls and Ethan Witwer. \emph{Maybenot: A Framework for Traffic Analysis Defenses}. WPES 2023.}
  \begin{center}
    \includegraphics<1>[width=.85\textwidth]{img/wpes-table}%
  \end{center}
\end{frame}

\section{Wrap-up}
\begin{frame}{Challenges}
  \framesubtitle{simplicity vs. expressiveness}
  \begin{itemize}
    \item some defenses aren't designed for state machine models
    \begin{itemize}
        \item we can only approximate their behavior
        \item large machines may be needed to do so
    \end{itemize}
    \item specific challenges in implementing defenses
    \begin{itemize}
        \item no event for queued cells
        \item no clean mechanism for counting
        \item no precise way to measure time
    \end{itemize}
    \item new features added in v2 of the framework
  \end{itemize}
\end{frame}

\begin{frame}{Takeaways}
  \framesubtitle{``you don't know the power of the dark side''}
  \begin{itemize}
    \item \url{https://crates.io/crates/{maybenot, maybenot-simulator}}
    \item \url{https://github.com/ewitwer/maybenot-defenses}
  \end{itemize}
\end{frame}

  \backmatter

\end{document}
